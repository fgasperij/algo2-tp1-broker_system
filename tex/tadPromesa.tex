\section{TAD \tadNombre{Promesa}}

\begin{tad}{\tadNombre{Promesa}}
\tadIgualdadObservacional{c}{c'}{promesa}{}

\tadParametrosFormales{
    \tadEncabezadoInline{g\'eneros}{}
}

\tadGeneros{}
\tadExporta{}
\tadUsa{\tadNombre{Bool, Nat}}

\tadAlinearFunciones{cantidad }{l\'imite, t\'itulo, bool, nat ,cliente}

%----------------------------------------------------
% Observadores 
%----------------------------------------------------
\tadObservadores
\tadOperacion{l\'imite}{promesa}{nat}{}
%Habria que aclarar que nombre es renombre de string
%l?mite es nat, cant es nat

\tadOperacion{nombre}{promesa}{nombre}{}
\tadOperacion{vende?}{promesa}{bool}{}
\tadOperacion{cantidad}{promesa}{nat}{} 
% Lo agregamos para la igualdad observacional y poder distinguir
% Si por ejemplo, 2 personas tratan de comprar lo mismo, con las mismas condiciones
% El conjunto los absorveria en uno, y eso estaria mal
%\tadOperacion{cliente}{promesa}{nat}{} 

%----------------------------------------------------
% Generadores 
%----------------------------------------------------
\tadGeneradores
%Ese bool es para el observador vende?
\tadOperacion{nueva}{l\'imite, t\'itulo, bool, nat}{promesa}{}
%----------------------------------------------------
% AXIOMAS
%----------------------------------------------------
\tadAxiomas[\paratodo{nat}{l, c}, \paratodo{bool}{v}, \paratodo{titulo}{t}]
\tadAlinearAxiomas{cantidad(nueva($l$,$t$,$v$,$c$))}
\tadAxioma{l\'imite(nueva($l$,$t$,$v$,$c$))}{$l$}
\tadAxioma{t\'itulo(nueva($l$,$t$,$v$,$c$))}{$t$}
\tadAxioma{vende?(nueva($l$,$t$,$v$,$c$))}{$v$}
\tadAxioma{cantidad(nueva($l$,$t$,$v$,$c$))}{$c$}
%\tadAxioma{cliente(nueva($l$,$t$,$v$,$a$,$c$))}{$c$}

\end{tad}