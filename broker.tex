\documentclass[10pt, a4paper]{article}
\usepackage[paper=a4paper, left=1.5cm, right=1.5cm, bottom=1.5cm, top=3.5cm]{geometry}
\usepackage[latin1]{inputenc}
\usepackage[T1]{fontenc}
\usepackage[spanish]{babel}
\usepackage{indentfirst}
\usepackage{fancyhdr}
\usepackage{latexsym}
\usepackage{lastpage}
\usepackage{aed2-symb,aed2-itef,aed2-tad}
\usepackage[colorlinks=true, linkcolor=blue]{hyperref}
\usepackage{calc}

\newcommand{\f}[1]{\text{#1}}
\renewcommand{\paratodo}[2]{$\forall~#2$: #1}
probando tags
\sloppy

%El siguiente paquete permite escribir la caratula facilmente
\usepackage{caratula}
\hypersetup{
  pdftitle={ Algoritmos y estructuras de datos II - TP1 },
  colorlinks,
  citecolor=black,
  filecolor=black,
  linkcolor=black,
  urlcolor=black 
}

%Datos para la caratula
\materia{Algoritmos y Estructuras de Datos II}

\titulo{TP 1 - Broker System}

%\fecha{DDDD dd de MMMM, AAAA}

\grupo{Grupo 20}

\integrante{Fernando Gasperi Jabalera}{56/09}{fgasperijabalera@gmail.com}
\integrante{Esteban Romero}{}{}
\integrante{Leandro Tozzi}{-}{leandro.tozzi@gmail.com}
\integrante{Alfredo Terrile Cendoya}{}{}

\parskip=5pt % 10pt es el tama�o de fuente

% Pongo en 0 la distancia extra entre ?temes.
\let\olditemize\itemize
\def\itemize{\olditemize\itemsep=0pt}

% Acomodo fancyhdr.
\pagestyle{fancy}
\thispagestyle{fancy}
\addtolength{\headheight}{1pt}
\lhead{Algoritmos y Estructuras de Datos II}
\rhead{$1^{\mathrm{er}}$ cuatrimestre de 2014}
\cfoot{\thepage /\pageref{LastPage}}
\renewcommand{\footrulewidth}{0.4pt}

\author{Algoritmos y Estructuras de Datos II, DC, UBA.}
\date{}
\title{TP 1: Especificaci\'on \\ Broker System \\ Grupo: 20}


\begin{document}

%Pagina de titulo e indice
\thispagestyle{empty}

\maketitle
\tableofcontents

\newpage

% decisiones y observaciones
\section{Desarrollo}

En este lugar vamos agregando todas las decisiones y observaciones que nos parezcan pertinentes documentar.
%TADS
\section{TAD \tadNombre{Broker}}

\begin{tad}{\tadNombre{Broker}}
\tadIgualdadObservacional{b}{b'}{broker}{titulos($b$) $\igobs$ titulos($b'$) $\land$ clientes($b$) $\igobs$ clientes($b'$) $\land$ $(\forall c:cliente)$ (esta?($c$, clientes($b$)) $\igobs$ esta?($c$, clientes($b'$)) $\land$ esta?($c$, clientes($b$)) $\implies$ (promesas($c$, $b$) $\igobs$ promesas($c$, $b'$) $\land$ acciones($c$, $b$) $\igobs$ acciones($c$, $b'$)))}

\tadGeneros{broker}
\tadExporta{broker, observadores, enAlza}
\tadUsa{\tadNombre{Bool, Nat}}

\tadAlinearFunciones{CambiarCotizacion}{nombre /n, cotizacion, broker /b}

\tadObservadores
\tadOperacion{titulos}{broker}{conj(titulo)}{}
\tadOperacion{promesas}{cliente /c, broker}{conj(promesas)}{$c$ $\in$ clientes($b$)}
\tadOperacion{clientes}{broker}{conj(clientes)}{}
\tadOperacion{acciones}{cliente /c,nombre /n, broker /b}{nat}{estaDefinido?($n$, titulos($b$)) $\land$ $c$ $\in$ clientes($b$)} 

%----------------------------------------------------
% Generadores 
%----------------------------------------------------
\tadGeneradores
\tadOperacion{nuevo}{conj(titulo) /ts, conj(cliente) /cs}{broker}{$\neg$Vacio?($cs$) $\land$ $\neg$hayNombresRepetidos($ts$)}
\tadOperacion{CambiarCotizacion}{nombre /n, cotizacion, broker /b}{broker}{estaDefinido?($n$, titulos($b$))}
\tadOperacion{AgPromesa}{cliente /c, promesa /p, broker /b}{broker}{$c$ $\in$ clientes($b$) $\land$ estaDefinido?(
nombreTitulo(p),titulos($b$)) $ \yluego$ 
cantidad($p$) $\leq$ maximoAcciones(nombreTitulo($p$), titulos($b$)) $\land$ $\neg$HayPromMismoTipo($p$, promesas($c$, $b$))) $ \yluego$ (vende?($p$) $\implies$ puedeVender?($c$, $p$, $b$))}

%----------------------------------------------------
% Otras Operaciones
%----------------------------------------------------
\tadOtrasOperaciones
\tadAlinearFunciones{accDspDeComprasDeOtros}{cliente /c, conj(promesa) /ps, nombre, cotizacion, nat}
% Revisar: enAlza es observador de titulo. Acá está mal la aridad
\tadOperacion{hayNombresRepetidos}{conj(titulo) /ts}{bool}{}
\tadOperacion{maximoAcciones}{nombre /n, conj(titulo) /ts}{nat}{}
\tadOperacion{enAlza}{nombre /n, broker /b}{bool}{estaDefinido?($n$, titulos($b$))}
\tadOperacion{HayPromMismoTipo}{promesa /p, conj(promesa) /ps}{bool}{}
\tadOperacion{cumplirPromesas}{cliente /c, conj(promesa) /ps, nombre, cotizacion, nat, broker /b}{conj(promesa)}{}
\tadOperacion{dameMaximo}{nombre,conj(titulo)}{nat}{}
\tadOperacion{accionesLibres}{titulo /t, broker /b}{nat}{} %Drive dice que devuelve acciones
\tadOperacion{accionesTomadas}{nombre, conj(cliente), broker}{nat}{}%Drive dice que devuelve acciones
\tadOperacion{cumplirVentas}{conj(promesa), nombre, cotizacion}{conj(promesa)}{}
\tadOperacion{existeVenta}{promesa, nombre, cotizacion}{bool}{}
\tadOperacion{accLibresDspVenta}{conj(cliente),nombre,cotizacion,nat,broker}{nat}{}
\tadOperacion{vendioAcciones}{conj(promesa),nombre,cotizacion}{bool}{}
\tadOperacion{nuevasAccLibres}{conj(promesa),nombre,cotizacion}{nat}{}
\tadOperacion{cumplirCompras}{cliente, conj(promesa),titulo, cotizacion, conj(clientes), acciones, broker}{conj(promesas)}{}
\tadOperacion{accDspDeComprasDeOtros}{conj(promesa),nombre,cotizacion,nat}{nat}{}
\tadOperacion{promDspDeVerSiCompro}{conj(promesa),nombre,cotizacion,nat}{conj(promesa)}{}
\tadOperacion{puedeVender?}{cliente /c, promesa /p, broker /b}{bool}{}
\tadOperacion{estaDefinido?}{nombre, conj(titulo)}{bool}{}
\tadOperacion{seCumple?}{promesa, broker}{bool}{}
\tadOperacion{dameCotizacion}{nombre, conj(titulo)}{nat}{}
\tadOperacion{tituloNuevaCot}{nombre, cotizacion, conj(titulo)}{conj(titulo)}{}
\tadOperacion{accClienteDspVenta}{nat,conj(promesa), nombre, cotizacion}{nat}{}
\tadOperacion{accDspDeCompras}{cliente, nombre, conj(promesas), conj(cliente), nat, nat, broker}{nat}{}
\tadOperacion{accClienteDspCompras}{conj(promesa),nombre,cotizacion,nat,nat}{nat}{}



\newpage
\tadAxiomas [\paratodo{broker}{b}, \paratodo{cliente}{c, c1}, \paratodo{conj(cliente)}{cs}, \paratodo{conj(promesa)}{ps}, \paratodo{nat}{a, av, ct}, \paratodo{conj(titulo)}{ts}, \paratodo{string}{n}, \paratodo{titulo}{t}]
\tadAlinearAxiomas{accDspDeComprasDeOtros($ps$, $n$, $ct$, $a$)}
%---------------------------------------------------------------------
%Axiomas de Promesas
%---------------------------------------------------------------------
\tadAxioma{promesas($c$, nuevo($ts$, $cs$))}{$\emptyset$}
\tadAxioma{promesas($c$, AgPromesa($c_1$, $p$, $b$))}{\IF seCumple?($p$, $b$) THEN promesas($c$, $b$) ELSE {\IF $c = c_1$ THEN Ag(p, promesas($c$, $b$))
ELSE promesas($c$, $b$) FI}FI}
\tadAxioma{promesas($c$, cambiarCot($n$, $ct$, $b$))}{cumplirPromesas($c$, promesas($c$, $b$), $n$, $ct$, accionesLibres($n$, $b$), $b$)}
%---------------------------------------------------------------------
%Axiomas de Titulos
%---------------------------------------------------------------------
\tadAxioma{titulos(nuevo($ts$, $cs$))}{$ts$}
\tadAxioma{titulos(AgTitulo($t$, $b$))}{Ag($t$, titulos($b$))}
\tadAxioma{titulos(cambiarCot($n$, $ct$, $b$))}{tituloNuevaCot($n$, $ct$, titulos($b$))}

%---------------------------------------------------------------------
%Axiomas de Clientes
%---------------------------------------------------------------------
\tadAxioma{clientes(nuevo($ts$, $cs$))}{$cs$}
\tadAxioma{clientes(cambiarCot($n$, $ct$, $b$))}{clientes($b$)}
\tadAxioma{clientes(AgPromesa($c$, $p$, $b$))}{clientes($b$)}
\tadAxioma{clientes(AgTitulo($t$, $b$))}{clientes($b$)}

%---------------------------------------------------------------------
%Axiomas de acciones
%---------------------------------------------------------------------
\tadAxioma{acciones($c$, $t$,nuevo($ts$, $cs$))}{0}
\tadAxioma{acciones($c$, $t$, CambiarCot($n$, $ct$, $b$))}{\IF $t = n$ THEN accDspDeCompras($c$, $n$, promesas($c$, $b$), clientes($b$),\\
 accClienteDspVenta(acciones($c$, $t$, $b$), promesas($c$, $b$), $n$, $ct$),\\ accLibresDspVenta(clientes($b$), $n$, $ct$, accionesLibres($n$, $b$), $b$), $b$) ELSE acciones($c$, $t$, $b$) FI}
\tadAxioma{acciones($c$, $n$, AgPromesa($c_1$, $p$, $b$))}{\IF $c = c_1$ $\land$ seCumple?($p$, $b$) THEN {\IF vende?($p$) THEN acciones($c$, $n$, $b$) - cantidad($p$) ELSE acciones($c$, $n$, $b$) + cantidad($p$) FI } ELSE acciones($c$, $n$, $b$) FI }


%---------------------------------------------------------------------
%Axiomas Otras Operaciones
%---------------------------------------------------------------------

%---------------------------------------------------------------------
%Axiomas de hayNombresRepetidos
%---------------------------------------------------------------------
\tadAxioma{hayNombresRepetidos?($ts$)}{
\IF vacio?($ts$) THEN
  false
ELSE {\IF estaDefinido(nombre(dameUno($ts$)), sinUno($ts$)) THEN true ELSE hayNombresRepetidos(sinUno($ts$)) FI}
FI}
%---------------------------------------------------------------------
%Axiomas de maximoAcciones
%---------------------------------------------------------------------
\tadAxioma{maximoAcciones($n$, $ts$)}{
\IF nombre(dameUno($ts$)) = $n$ THEN
  maximo(dameUno($ts$)
ELSE maximoAcciones($n$, sinUno($ts$))
FI}
%---------------------------------------------------------------------
% Otras Funciones que se utilizan en las restricciones de AgPromesa
%---------------------------------------------------------------------
\tadAxioma{puedeVender?($c$, $p$, $b$)}{cantidad($p$) $\leq$ acciones($c$, nombre($p$), $b$)}
\tadAxioma{HayPromMismoTipo($p$, $ps$)}{\IF vacio?($ps$) THEN false ELSE {\IF vende?(dameUno($ps$)) = vende?($p$) $\land$ \\ nombre(dameUno($ps$)) = nombre($p$) THEN true ELSE HayPromMismoTipo($p$, sinUno($ps$))FI} FI}
\tadAxioma{estaDefinido?($n$, $ts$)}{\IF vacio?($ts$) THEN false ELSE {\IF nombre(dameUno($ts$)) = $n$ THEN true ELSE 
estaDefinido?($n$, sinUno($ts$)) FI}FI}
%--------------------------------------------------------------------

\tadAxioma{seCumple?($p$, $b$)}{
\IF vende?($p$) 
  THEN dameCotizacion(nombreTitulo($p$), titulos($b$)) $\textless$ limite($p$) 
ELSE 
  {\IF accionesLibres(nombreTitulo($p$), $b$) > cantidad($p$) 
    THEN dameCotizacion(nombreTitulo($p$), titulos($b$)) $\textgreater$ limite($p$) 
  ELSE false 
  FI} 
FI}

\tadAxioma{dameCotizacion($n$, $ts$)}{\IF nombre(dameUno($ts$)) = $n$ THEN cotizacion(dameUno($ts$)) ELSE dameCotizacion($n$,sinUno($ts$)) FI}

\tadAxioma{accionesLibres($n$, $b$)}{dameMaximo($n$, titulos($b$)) - accionesTomadas($n$, clientes($b$), $b$)}

\tadAxioma{dameMaximo($n$, $ts$)}{\IF nombre(dameUno($ts$)) = $n$ THEN maximo(dameUno($ts$)) ELSE dameMaximo($n$,sinUno($ts$)) FI}

\tadAxioma{accionesTomadas($n$, $cs$, $b$)}{\IF vacio?($cs$) THEN 0 ELSE acciones(dameUno($cs$), $n$, $b$) + accionesTomadas($n$, sinUno($cs$), $b$) FI}

\tadAxioma{cumplirPromesas($c$, $ps$, $t$, $ct$, $a$, $b$)}{cumplirCompras($c$, cumplirVentas($ps$, $t$, $ct$), $t$, $ct$, clientes($b$),\\ accLibresDespVenta(clientes($b$), $t$, $ct$, $a$, $b$), $b$)}

% Duda: ct va entre {} ?
\tadAxioma{cumplirVentas($ps$, $n$, $ct$)}{\IF vacio?($ps$) THEN $\emptyset$ ELSE {\IF existeVenta(dameUno($ps$), $n$, $ct$) THEN sinUno($ps$) ELSE Ag(dameUno($ps$), cumplirVentas(sinUno($ps$), $n$, $ct$)) FI} FI}

\tadAxioma{existeVenta($p$, $n$, $ct$)}{vende?($p$) $\land$ nombre($p$) = $n$ $\land$ $ct$ $\textless$ limite($p$)}

\tadAxioma{accLibresDspVenta($cs$, $n$, $ct$, $a$, $b$)}{\IF vacio?($cs$) THEN $a$ ELSE {\IF vendioAcciones(promesas(dameUno($cs$), $b$), $n$, $ct$) THEN 
nuevasAccLibres(promesas(dameUno($cs$), $b$), $n$, $ct$) +\\ accLibresDspVenta(sinUno($cs$),$n$,$ct$,$a$,$b$) ELSE accLibresDspVenta($p$,sinUno($cs$),$n$,$c$,$a$) FI} FI}

\tadAxioma{vendioAcciones($ps$, $n$, $ct$)}{\IF vacio($ps$) THEN false ELSE {\IF vende?(dameUno($p$)) $\land$ nombre(dameUno($ps$)) = $n$ $\land$ \\ 
limite(dameUno($ps$)) $\textless$ $ct$ THEN true ELSE vendioAcciones(sinUno($ps$),$n$,$ct$) FI} FI}

\tadAxioma{nuevasAccLibres($ps$, $n$, $ct$)}{\IF vacio($ps$) THEN 0 ELSE {\IF vende?(dameUno($ps$)) $\land$ nombre(dameUno($ps$)) = $n$ $\land$ \\ 
limite(dameUno($ps$)) $\textless$ $ct$ THEN cantidad(dameUno($ps$)) ELSE nuevasAccLibres(sinUno($p$), $n$, $ct$) FI} FI}

\tadAxioma{cumplirCompras($c$, $ps$, $t$, $ct$, $cs$, $a$, $b$)}{\IF dameUno($cs$) = $c$ THEN promDspDeVerSiCompro($ps$, $t$, $ct$, $a$) ELSE 
cumplirCompras($c$, $ps$, $t$, $ct$, sinUno($cs$),\\ accDspDeComprasDeOtros(promesas(dameUno($cs$), $b$), $t$, $ct$, $a$), $b$) FI}

\tadAxioma{accDspDeComprasDeOtros($ps$, $n$, $ct$, $a$)}{\IF vacio?($ps$) THEN $a$ ELSE {\IF $\neg$vende?(dameUno($ps$)) $\land$ nombre(dameUno($ps$)) = $n$ $\land$ 
limite(dameUno($ps$)) $\textless$ $ct$ THEN $a$ - cantidad($p$) ELSE accDspDeComprasDeOtros(sinUno($p$), $n$, $ct$, $a$) FI} FI}

\tadAxioma{promDspDeVerSiCompro($ps$, $n$, $ct$, $a$)}{\IF vacio?($ps$) THEN $\emptyset$ ELSE {\IF $\neg$vende?(dameUno($ps$)) $\land$ nombre(dameUno($ps$)) = $n$ $\land$ 
limite(dameUno($ps$)) $\textless$ $ct$ $\land$  $a \geq$ cantidad(dameUno($ps$)) THEN sinUno($ps$) ELSE 
Ag(dameUno($ps$), promDspDeVerSiCompro(sinUno($ps$), $n$, $ct$, $a$)) FI} FI}

% TituloNuevaCot 
%-----------------
%Se usa en la axiomatizacion del observador titulos
\tadAxioma{tituloNuevaCot($n$, $ct$, $ts$)}{\IF nombre(dameUno($ts$)) = $n$ THEN Ag(cambiarValor($ct$,dameUno($ts$)),sinUno($ts$)) ELSE Ag(dameUno($ts$),tituloNuevaCot($n$,$ct$,sinUno($ts$)) FI}

% Funciones utilizadas en axiomatizacion del obs Acciones
%----------------------------------------------------------
% AccionesClienteDspVenta
\tadAxioma{accClienteDspVenta($a$, $ps$, $n$, $ct$)}{\IF vacio?($ps$) THEN $a$ ELSE {\IF existeVenta(dameUno($ps$), $t$, $ct$) THEN $a$ - cantidad(dameUno($ps$)) ELSE accClienteDspVenta($a$, sinUno($ps$), $n$, $ct$) FI} FI}

\tadAxioma{accDspDeCompras($c$, $n$, $ps$, $cs$, $a$, $av$, $b$)}{\IF dameUno($cs$) = $c$ THEN 
accClienteDspCompras($ps$, $t$, $ct$, $a$, $av$) ELSE accDspDeCompras($c$, $n$, $ps$, sinUno($cs$), $ct$,
accDspDeComprasDeOtros(promesas(dameUno($cs$), $b$), $t$, $ct$, $av$), $b$) FI}

\tadAxioma{accClienteDspCompras($ps$, $n$, $ct$, $a$, $av$)}{\IF vacio($ps$) THEN $a$ ELSE 
{\IF $\neg$vende?(dameUno($ps$)) $\land$ nombre(dameUno($ps$)) = $n$ $\land$ limite(dameUno($ps$)) $\textless$ $c$
$\land$ $av$ $\leq$ cantidad(dameUno($ps$)) THEN $a$ + cantidad(dameUno($ps$)) ELSE promDspDeVerSiCompro(sinUno($ps$), $n$, $ct$, $a$, $av$)FI} FI}


\end{tad}
\section{TAD \tadNombre{Promesa}}

\begin{tad}{\tadNombre{Promesa}}
\tadIgualdadObservacional{p}{p'}{promesa}{$nombreTitulo(p) \igobs nombreTitulo(p') \land vende?(p) \igobs vende?(p') \land cantidad(p) \igobs cantidad(p') \land limite(p) \igobs limite(p')$}

\tadGeneros{promesa}
\tadExporta{observadores}
\tadUsa{\tadNombre{Bool, Nat}}

\tadAlinearFunciones{nombreT\'itulo}{l\'imite, t\'itulo, bool, nat ,cliente}

%----------------------------------------------------
% Observadores 
%----------------------------------------------------
\tadObservadores
\tadOperacion{l\'imite}{promesa}{nat}{}
\tadOperacion{nombreT\'itulo}{promesa}{nombre}{}
\tadOperacion{vende?}{promesa}{bool}{}
\tadOperacion{cantidad}{promesa}{nat}{} 

%----------------------------------------------------
% Generadores 
%----------------------------------------------------
\tadGeneradores
\tadOperacion{nueva}{l\'imite, t\'itulo, bool, nat}{promesa}{}
%----------------------------------------------------
% AXIOMAS
%----------------------------------------------------
\tadAxiomas[\paratodo{nat}{l, c}, \paratodo{bool}{v}, \paratodo{titulo}{t}]
\tadAlinearAxiomas{nombreT\'itulo(nueva($l$,$t$,$v$,$c$))}
\tadAxioma{l\'imite(nueva($l$,$t$,$v$,$c$))}{$l$}
\tadAxioma{nombreT\'itulo(nueva($l$,$t$,$v$,$c$))}{$t$}
\tadAxioma{vende?(nueva($l$,$t$,$v$,$c$))}{$v$}
\tadAxioma{cantidad(nueva($l$,$t$,$v$,$c$))}{$c$}

\end{tad}
\section{TAD \tadNombre{T\'itulo}}

\begin{tad}{\tadNombre{T\'itulo}}
\tadIgualdadObservacional{c}{c'}{t\'itulo}{}

\tadParametrosFormales{
    \tadEncabezadoInline{g\'eneros}{}
}

\tadGeneros{}
\tadExporta{}
\tadUsa{\tadNombre{Bool, Nat}}

\tadAlinearFunciones{CambiarValor}{nombre, cotizaci\'on, m\'aximo}

%----------------------------------------------------
% Observadores 
%----------------------------------------------------
\tadObservadores
%Hay que aclarar que: Nombre es renombre de string
\tadOperacion{nombre}{t\'itulo}{string}{}
\tadOperacion{m\'aximo}{t\'itulo}{nat}{}
\tadOperacion{cotizaci\'on}{t\'itulo}{nat}{}
\tadOperacion{enAlza}{t\'itulo}{nat}{}
%----------------------------------------------------
% Generadores 
%----------------------------------------------------
\tadGeneradores
\tadOperacion{nuevo}{nombre, cotizaci\'on, m\'aximo}{t\'itulo}{}
\tadOperacion{cambiarValor}{valor, t\'itulo}{t\'itulo}{}
%----------------------------------------------------
% AXIOMAS
%----------------------------------------------------
\tadAxiomas %[\paratodo{broker}{c, d}, \paratodo{$\alpha$}{a, b}]
\tadAlinearAxiomas{cotizaci\'on(cambiarValor($v$,$t$))}

\tadAxioma{nombre(nueva($n$,$m$,$c$)))}{$n$}
\tadAxioma{m\'aximo(nueva($n$,$m$,$c$)))}{$m$}
\tadAxioma{cotizaci\'on(nueva($n$,$m$,$c$))}{$c$}
\tadAxioma{cotizaci\'on(cambiarValor($v$,$t$))}{$v$}
\tadAxioma{enAlza(nueva($n$,$m$,$c$))}{true}
\tadAxioma{enAlza(cambiarValor($v$,$t$))}{cotizaci\'on($t$) $\textless$ $v$}

\end{tad}

\end{document}
