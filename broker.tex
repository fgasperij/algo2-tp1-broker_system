\documentclass[10pt, a4paper]{article}
\usepackage[paper=a4paper, left=1.5cm, right=1.5cm, bottom=1.5cm, top=3.5cm]{geometry}
\usepackage[latin1]{inputenc}
\usepackage[T1]{fontenc}
\usepackage[spanish]{babel}
\usepackage{indentfirst}
\usepackage{fancyhdr}
\usepackage{latexsym}
\usepackage{lastpage}
\usepackage{aed2-symb,aed2-itef,aed2-tad}
\usepackage[colorlinks=true, linkcolor=blue]{hyperref}
\usepackage{calc}

\newcommand{\f}[1]{\text{#1}}
\renewcommand{\paratodo}[2]{$\forall~#2$: #1}

\sloppy

\hypersetup{%
 % Para que el PDF se abra a pagina completa.
 pdfstartview= {FitH \hypercalcbp{\paperheight-\topmargin-1in-\headheight}},
 pdfauthor={C?tedra de Algoritmos y Estructuras de Datos II - DC - UBA},
 pdfkeywords={TADs b?sicos},
 pdfsubject={Tipos abstractos de datos b?sicos}
}

\parskip=5pt % 10pt es el tama�o de fuente

% Pongo en 0 la distancia extra entre ?temes.
\let\olditemize\itemize
\def\itemize{\olditemize\itemsep=0pt}

% Acomodo fancyhdr.
\pagestyle{fancy}
\thispagestyle{fancy}
\addtolength{\headheight}{1pt}
\lhead{Algoritmos y Estructuras de Datos II}
\rhead{$1^{\mathrm{er}}$ cuatrimestre de 2014}
\cfoot{\thepage /\pageref{LastPage}}
\renewcommand{\footrulewidth}{0.4pt}

\author{Algoritmos y Estructuras de Datos II, DC, UBA.}
\date{}
\title{TP 1: Especificaci\'on \\ Broker System \\ Grupo:20}


\begin{document}

%Pagina de titulo e indice
\thispagestyle{empty}

\maketitle
\tableofcontents

\newpage

%TADS

\section{TAD \tadNombre{Broker}}
\begin{tad}{\tadNombre{Broker}}
\tadIgualdadObservacional{c}{c'}{broker}{}

\tadParametrosFormales{
    \tadEncabezadoInline{g\'eneros}{}
}

\tadGeneros{}
\tadExporta{}
\tadUsa{\tadNombre{Bool, Nat}}

\tadAlinearFunciones{CambiarCotizacion}{nombre /n, cotizaci\'on, broker /b}

\tadObservadores
\tadOperacion{t\'itulos}{broker}{conj(t\'itulo)}{}
\tadOperacion{promesas}{cliente, broker}{conj(promesas)}{}
\tadOperacion{clientes}{broker}{conj(clientes)}{}
% Aca en el drive devuelve "cantidad"
\tadOperacion{acciones}{cliente /c,nombre /n, broker /b}{nat}{est\'aDefinido($n$,t\'itulos($b$)) $\land$ $c$ $\in$ clientes($b$)} 

%----------------------------------------------------
% Generadores 
%----------------------------------------------------
\tadGeneradores
\tadOperacion{nuevo}{conj(t\'itulo) /ts, conj(cliente) /cs}{broker}{$\neg$Vac\'io?($cs$) $\land$ noHayNombresRepetidos($ts$)}
\tadOperacion{CambiarCotizaci\'on}{nombre /n, cotizaci\'on, broker /b}{broker}{est\'aDefinido($n$,t\'itulos($b$))}
\tadOperacion{AgPromesa}{cliente /c, promesa /p, broker /b}{broker}{$c$ $\in$ clientes($b$) $\land$ est\'aDefinido(
nombre(p),t\'itulos($b$)) $ \yluego$ 
cantidad($p$) $\leq$ m\'aximo(nombre($p$), $c$) $\land$ $\neg$HayPromMismoTipo(nombre($p$),promesas($c$,$b$))) $ \yluego$ (vende?($p$) $\implies$ puedeVender?($c$,$p$,$b$))}
\tadOperacion{AgT\'itulo}{t\'itulo /t,broker /b}{broker}{$\neg$($t$ $\in$ t\'itulos($t$))}

%----------------------------------------------------
% Otras Operaciones
%----------------------------------------------------
\tadOtrasOperaciones
\tadAlinearFunciones{accDpsDeComprasDeOtros}{cliente /c, conj(promesa) /ps, nombre, cotizaci\'on, nat}
\tadOperacion{enAlza}{nombre /n, broker /b}{bool}{est\'aDefinido($n$,t\'itulos($b$))}
\tadOperacion{HayPromMismoTipo}{promesa /p, conj(promesa) /ps, t\'itulo /t}{bool}{}
\tadOperacion{cumplirPromesas}{cliente /c, conj(promesa) /ps, nombre, cotizaci\'on, nat, broker /b}{conj(promesa)}{}
\tadOperacion{dameM\'aximo}{nombre,conj(t\'itulo)}{nat}{}
\tadOperacion{accionesLibres}{t\'itulo /t, broker /b}{nat}{} %Drive dice que devuelve acciones
\tadOperacion{accionesTomadas}{nombre, conj(cliente), broker}{nat}{}%Drive dice que devuelve acciones
\tadOperacion{cumplirVentas}{conj(promesa), nombre, cotizaci\'on}{conj(promesa)}{}
\tadOperacion{ExisteVenta}{promesa, nombre, cotizaci\'on}{bool}{}
\tadOperacion{accLibresDpsVenta}{conj(cliente),nombre,cotizaci\'on,nat,broker}{nat}{}
\tadOperacion{Vendi\'oAcciones}{conj(promesa),nombre,cotizaci\'on}{bool}{}
\tadOperacion{nuevasAccLibres}{conj(promesa),nombre,cotizaci\'on}{nat}{}
\tadOperacion{cumplirCompras}{cliente, conj(promesa),t\'itulo, cotizaci\'on, conj(clientes), acciones, broker}{conj(promesas)}{}
% Revisar si esta bien el nombre en el resto del documento
\tadOperacion{accDspDeComprasDeOtros}{conj(promesa),nombre,cotizaci\'on,nat}{nat}{}
\tadOperacion{promDspDeVerSiCompro}{conj(promesa),nombre,cotizaci\'on,nat}{conj(promesa)}{}
\tadOperacion{puedeVender?}{cliente /c, promesa /p, broker /b}{bool}{}
\tadOperacion{est\'aDef?}{nombre, conj(t\'itulo)}{bool}{}


\newpage
\tadAxiomas %[\paratodo{broker}{c, d}, \paratodo{$\alpha$}{a, b}]
\tadAlinearAxiomas{accDspDeComprasDeOtros($ps$,$n$,$ct$,$a$)}
%---------------------------------------------------------------------
%Axiomas de Promesas
%---------------------------------------------------------------------
\tadAxioma{promesas($c$,nuevo($ts$,$cs$)}{$\emptyset$}
\tadAxioma{promesas($c$,AgPromesa($c_1$,$p$,$b$))}{\IF seCumple($p$,$b$) THEN promesas($c$,$b$) ELSE {\IF $c = c_1$ THEN Ag(p,promesas($c$,$b$))
ELSE promesas($c$,$b$) FI}FI}
\tadAxioma{promesas($c$,cambiarCot($n$,$ct$,$b$))}{cumplirPromesas($c$,promesas($c$,$b$),$n$,$ct$, accionesLibres($n$,$b$),$b$))}
%---------------------------------------------------------------------
%Axiomas de Titulos
%---------------------------------------------------------------------
\tadAxioma{t\'itulos(nuevo($ts$,$cs$))}{$ts$}
\tadAxioma{t\'itulos(AgT\'itulo($t$,$b$))}{Ag($t$,t\'itulos($b$))}
\tadAxioma{t\'itulos(cambiarCot($n$,$ct$,$b$))}{t\'ituloNuevaCot($n$,$ct$,t\'itulos($b$))}

%---------------------------------------------------------------------
%Axiomas de Clientes
%---------------------------------------------------------------------
\tadAxioma{clientes(nuevo($ts$,$cs$))}{$cs$}
\tadAxioma{clientes(cambiarCot($n$,$ct$,$b$))}{clientes($b$)}
\tadAxioma{clientes(AgPromesa($c$,$p$,$b$))}{clientes($b$)}
\tadAxioma{clientes(AgT\'itulo($t$,$b$))}{clientes($b$)}

%---------------------------------------------------------------------
%Axiomas de acciones
%---------------------------------------------------------------------
\tadAxioma{acciones($c$,$t$,nuevo($ts$,$cs$))}{0}
% Se le pasan muchos argumentos a la funcion afectando la est?tica
\tadAxioma{acciones($c$,$t$,CambiarCot($n$,$ct$,$b$))}{\IF $t = n$ THEN accionesDpsDeCompra($c$,$n$, promesas($c$,$t$,$b$), clientes($b$),\\
 accClienteDspVenta( acciones($c$,$t$,$b$), promesas($c$,$t$,$b$),$n$,$ct$),\\ accionesLibresDspVenta (clientes($b$),$n$,$ct$,accionesLibres($n$,$b$),$b$),$b$) ELSE acciones($c$,$t$,$b$) FI}


% REVISAR ESTA AXIOMATIZACION
\tadAxioma{acciones($c$,$n$,AgPromesa($c_1$,$p$,$b$))}{\IF $c = c_1$ $\land$ seCumple($p$,$b$) THEN {\IF vende?($p$) THEN acciones($c$,$n$,$b$) - cantidad($p$) ELSE acciones($c$,$n$,$b$) + cantidad($p$) FI } ELSE acciones($c$,$n$,$b$) FI }


%---------------------------------------------------------------------
%Axiomas Otras Operaciones
%---------------------------------------------------------------------
%---------------------------------------------------------------------
% Otras Funciones que se utilizan en las restricciones de AgPromesa
%---------------------------------------------------------------------
\tadAxioma{puedeVender?($c$,$p$,$b$)}{cantidad($p$) $\leq$ acciones($c$,nombre($p$),$b$)}
\tadAxioma{HayPromMismoTipo($p$,$ps$)}{\IF (vac\'io?($ps$) THEN false ELSE {\IF vende?(dameUno($ps$))=vende?($p$) $\land$ \\ nombre(dameUno($ps$))=nombre($p$) THEN true ELSE HayPromMismoTipo($p$,sinUno($ps$))FI} FI}
\tadAxioma{est\'aDefinido($n$,$ts$)}{\IF vac\'io?($ts$) THEN false ELSE {\IF nombre(dameUno($ts$))= $n$) THEN true ELSE 
est\'aDefinido($n$,sinUno($ts$)) FI}FI}
%--------------------------------------------------------------------

\tadAxioma{seCumple($p$,$b$)}{\IF vende?($p$) THEN dameCotizaci\'on(nombre($p$),t\'itulos($b$)) $\textless$ l\'imite($p$) ELSE dameCotizaci\'on(nombre($p$),t\'itulos($b$)) $\textgreater$ l\'imite($p$) FI }

\tadAxioma{dameCotizaci\'on($n$,$ts$)}{\IF nombre(dameUno($ts$)) = $n$ THEN cotizaci\'on(dameUno($ts$)) ELSE dameCotizaci\'on($n$,sinUno($ts$)) FI}

\tadAxioma{accionesLibres($n$,$b$)}{dameM\'aximo($n$,t\'itulos($b$)) - accionesTomadas($n$,clientes($b$),b)}

\tadAxioma{dameM\'aximo($n$,$ts$)}{\IF nombre(dameUno($ts$)) = $n$ THEN m\'aximo(dameUno($ts$)) ELSE dameM\'aximo($n$,sinUno($ts$)) FI}

\tadAxioma{accionesTomadas($n$,$cs$,$b$)}{\IF (vac\'io?($cs$) THEN 0 ELSE acciones(dameUno($cs$),$n$,$b$) + accionesTomadas($n$,sinUno($cs$),$b$) FI}

\tadAxioma{cumplirPromesas($c$,$ps$,$t$,$ct$,$a$,$b$)}{cumplirCompras($c$,cumplirVentas($ps$,$t$,$ct$), $t$, $ct$, clientes($b$),\\(accionesLibresDespVenta(clientes($b$),$t$,$ct$,$a$,$b$),$b$)}

% Duda: ct va entre {} ?
\tadAxioma{cumplirVentas($ps$,$n$,$ct$)}{\IF vac\'io?($ps$) THEN $\emptyset$ ELSE {\IF existeVenta(dameUno($ps$),$n$,$ct$) THEN $ps$ ELSE Ag(dameUno($ps$),cumplirVentas(sinUno($ps$),$n$,$ct$)) FI} FI}

\tadAxioma{existeVenta($p$,$n$,$ct$)}{vende?($p$) $\land$ nombre($p$) = $n$ $\land$ l\'imite($p$) $\textless$ $ct$}

\tadAxioma{accLibresDspVenta($cs$,$n$,$ct$,$a$,$b$)}{\IF vac\'io?($cs$) THEN $a$ ELSE {\IF vendi\'oAcciones(promesas(dameUno($cs$), $b$), $n$, $ct$) THEN 
nuevasAccionesLibres(promesas(dameUno($cs$),$b$), $n$,$c$) +\\ accionesLibresDspVenta(sinUno($cs$),$n$,$ct$,$a$,$b$) ELSE accionesLibresDspVenta($p$,sinUno($cs$),$n$,$c$,$a$) FI} FI}

\tadAxioma{vendi\'oAcciones($ps$,$n$,$ct$)}{\IF vac\'io($ps$) THEN false ELSE {\IF vende?(dameUno($p$)) $\land$ nombre(dameUno($ps$)) = $n$ $\land$ \\ 
l\'imite(dameUno($ps$)) $\textless$ $ct$ THEN true ELSE vendi\'oAcciones(sinUno($ps$),$n$,$ct$) FI} FI}

\tadAxioma{nuevasAccLibres($ps$,$n$,$c$)}{\IF vac\'io($ps$) THEN 0 ELSE {\IF vende?(dameUno($ps$)) $\land$ nombre(dameUno($ps$)) = $n$ $\land$ \\ 
l\'imite(dameUno($ps$)) $\textless$ $c$ THEN cantidad(dameUno($ps$)) ELSE nuevasAccionesLibres(sinUno($p$),$n$,$c$) FI} FI}

\tadAxioma{cumplirCompras($c$,$ps$,$t$,$ct$,$cs$,$a$,$b$)}{\IF dameUno($cs$) = $c$ THEN promDspDeVerSiCompro($ps$,$t$,$ct$,$a$) ELSE 
cumplirCompras($c$,$ps$,$t$,$ct$,sinUno($cs$),\\ accDspDeComprasDeOtros(promesas(dameUno($cs$),$b$),$t$,$ct$,$a$),$b$) FI}

\tadAxioma{accDspDeComprasDeOtros($ps$,$n$,$ct$,$a$)}{\IF vac\'io?($ps$) THEN $a$ ELSE {\IF vende?(dameUno($ps$)) $\land$ nombre(dameUno($ps$)) = $n$ $\land$ 
l\'imite(dameUno($ps$)) $\textless$ $c$ THEN $a$ - cantidad($p$) ELSE accDspDeComprasDeOtros(sinUno($p$),$n$,$ct$,$a$) FI} FI}

\tadAxioma{promDspDeVerSiCompro($ps$,$n$,$ct$,$a$)}{\IF vac\'io?($ps$) THEN $\emptyset$ ELSE {\IF $\neg$vende?(dameUno($ps$)) $\land$ nombre(dameUno($ps$)) = $n$ $\land$ 
l\'imite(dameUno($ps$)) $\textless$ $c$ $\land$  $a \geq$ cantidad(dameUno($p$)) THEN sinUno($ps$) ELSE 
Ag(dameUno($ps$),promDspDeVerSiCompro(sinUno($ps$),$n$,$ct$,$a$)) FI} FI}

% TituloNuevaCot 
%-----------------
%Se usa en la axiomatizacion del observador titulos
\tadAxioma{t\'ituloNuevaCot($n$,$ct$,$ts$)}{\IF nombre(dameUno($ts$)) = $n$ THEN Ag(cambiarValor($ct$,dameUno($ts$)),sinUno($ts$)) ELSE Ag(dameUno($ts$),t\'ituloNuevaCot($n$,$ct$,sinUno($ts$)) FI}

% Funciones utilizadas en axiomatizacion del obs Acciones
%----------------------------------------------------------
% AccionesClienteDspVenta
\tadAxioma{accClienteDspVenta($a$,$ps$,$n$,$ct$)}{\IF vac\'io?($ps$) THEN $a$ ELSE {\IF existeVenta(dameUno($ps$),$t$,$ct$) THEN $a$ - cantidad(dameUno($ps$)) ELSE accClienteDspVenta($a$,sinUno($ps$),$n$,$ct$) FI} FI}

\tadAxioma{accDspDeCompras($c$,$n$,$ps$,$cs$,$a$,$av$,$b$)}{\IF dameUno($cs$) = $c$ THEN 
accClienteDspCompra($ps$,$t$,$ct$,$a$,$av$) ELSE accDspDeCompras($c$,$n$,$ps$,sinUno($cs$),$ct$,
accDspDeComprasDeOtros(promesas(dameUno($cs$),$b$),$t$,$ct$,$av$),$b$) FI}

\tadAxioma{accClienteDspCompra($ps$,$n$,$ct$,$a$,$av$)}{\IF vac\'io($ps$) THEN $a$ ELSE 
{\IF $\neg$vende?(dameUno($ps$)) $\land$ nombre(dameUno($ps$)) = $n$ $\land$ l\'imite(dameUno($ps$)) $\textless$ $c$
$\land$ $av$ $\leq$ cantidad(dameUno($ps$)) THEN $a$ + cantidad(dameUno($ps$)) ELSE promDspDeVerSiCompro(sinUno($ps$), $n$,$ct$,$a$,$av$)FI} FI}


\end{tad}
\section{TAD \tadNombre{Promesa}}
\section{TAD \tadNombre{Promesa}}

\begin{tad}{\tadNombre{Promesa}}
\tadIgualdadObservacional{p}{p'}{promesa}{$nombreTitulo(p) \igobs nombreTitulo(p') \land vende?(p) \igobs vende?(p') \land cantidad(p) \igobs cantidad(p') \land limite(p) \igobs limite(p')$}

\tadGeneros{promesa}
\tadExporta{observadores}
\tadUsa{\tadNombre{Bool, Nat}}

\tadAlinearFunciones{nombreT\'itulo}{l\'imite, t\'itulo, bool, nat ,cliente}

%----------------------------------------------------
% Observadores 
%----------------------------------------------------
\tadObservadores
\tadOperacion{l\'imite}{promesa}{nat}{}
\tadOperacion{nombreT\'itulo}{promesa}{nombre}{}
\tadOperacion{vende?}{promesa}{bool}{}
\tadOperacion{cantidad}{promesa}{nat}{} 

%----------------------------------------------------
% Generadores 
%----------------------------------------------------
\tadGeneradores
\tadOperacion{nueva}{l\'imite, t\'itulo, bool, nat}{promesa}{}
%----------------------------------------------------
% AXIOMAS
%----------------------------------------------------
\tadAxiomas[\paratodo{nat}{l, c}, \paratodo{bool}{v}, \paratodo{titulo}{t}]
\tadAlinearAxiomas{nombreT\'itulo(nueva($l$,$t$,$v$,$c$))}
\tadAxioma{l\'imite(nueva($l$,$t$,$v$,$c$))}{$l$}
\tadAxioma{nombreT\'itulo(nueva($l$,$t$,$v$,$c$))}{$t$}
\tadAxioma{vende?(nueva($l$,$t$,$v$,$c$))}{$v$}
\tadAxioma{cantidad(nueva($l$,$t$,$v$,$c$))}{$c$}

\end{tad}
\section{TAD \tadNombre{T\'itulo}}
\section{TAD \tadNombre{T\'itulo}}

\begin{tad}{\tadNombre{T\'itulo}}
\tadIgualdadObservacional{t}{t'}{t\'itulo}{$nombre(t) \igobs nombre(t') \land maximo(t) \igobs maximo(t') \land cotizacion(t) \igobs cotizacion(t') \land enAlza(t) \igobs enAlza(t')$}

\tadGeneros{t\'itulo}
\tadExporta{t\'itulo, observadores}
\tadUsa{\tadNombre{Bool, Nat}}

\tadAlinearFunciones{CambiarValor}{nombre, cotizaci\'on, m\'aximo}

%----------------------------------------------------
% Observadores 
%----------------------------------------------------
\tadObservadores
%Hay que aclarar que: Nombre es renombre de string
\tadOperacion{nombre}{t\'itulo}{string}{}
\tadOperacion{m\'aximo}{t\'itulo}{nat}{}
\tadOperacion{cotizaci\'on}{t\'itulo}{nat}{}
\tadOperacion{enAlza}{t\'itulo}{nat}{}
%----------------------------------------------------
% Generadores 
%----------------------------------------------------
\tadGeneradores
\tadOperacion{nuevo}{nombre, cotizaci\'on, m\'aximo}{t\'itulo}{}
\tadOperacion{cambiarValor}{valor, t\'itulo}{t\'itulo}{}
%----------------------------------------------------
% AXIOMAS
%----------------------------------------------------
\tadAxiomas [\paratodo{nat}{c, m, v}, \paratodo{string}{n}, \paratodo{titulo}{t}]
\tadAlinearAxiomas{cotizaci\'on(cambiarValor($v$,$t$))}

%% Observadores
% nombre
\tadAxioma{nombre(nuevo($n$,$m$,$c$)))}{$n$}
\tadAxioma{nombre(cambiarValor($v$,$t$)))}{nombre($t$)}
% maximo
\tadAxioma{m\'aximo(nuevo($n$,$m$,$c$)))}{$m$}
\tadAxioma{m\'aximo(cambiarValor($v$,$t$)))}{nombre($t$)}
% cotizacion
\tadAxioma{cotizaci\'on(nuevo($n$,$m$,$c$))}{$c$}
\tadAxioma{cotizaci\'on(cambiarValor($v$,$t$))}{$v$}
% en alza
\tadAxioma{enAlza(nuevo($n$,$m$,$c$))}{true}
\tadAxioma{enAlza(cambiarValor($v$,$t$))}{cotizaci\'on($t$) $\textless$ $v$}

\end{tad}

\end{document}
